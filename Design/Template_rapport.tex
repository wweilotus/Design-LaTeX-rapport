\documentclass[a4paper, 11pt,twoside=false]{scrartcl} % Remplace la classe "article", correspond au standard européen.
\usepackage{Rapport2}
\usepackage{ctex}
\usepackage{indentfirst}
\setCJKmainfont{思源宋体}
%\usepackage[toc]{multitoc} double table des matières si nécessaire
\usepackage{lipsum}
\clearpairofpagestyles


%%% Paramètres pour les footers et la page de titre
\newcommand{\lab}{计算书}
\newcommand{\tit}{电动力绳弹射装置设计}
\newcommand{\nom}{}
\newcommand{\superv}{}
\newcommand{\depa}{}

\input{headtitre.tex}



\begin{document}


%%%%TITRE

\maketitle

\tableofcontents % table des matières
\newpage

%%%%%%% Début du document

\section{迁移}
她叫林徽因,出生于杭州,是许多人梦中期待的白莲。她在雨雾之都伦敦,
发生过一场空前绝后的康桥之恋。她爱过三个男子,爱得清醒,也爱得平静。徐
志摩为她徜徉在康桥,深情地等待一场旧梦可以归来。梁思成与她携手走过千山
万水,为完成使命而相约白头。金岳霖为她终身不娶,痴心不改地守候一世。可
她懂得人生飘忽不定,要学会随遇而安。真正的平静,不是避开车马喧嚣,而是
在心中修篱种菊。尽管如流往事,每一天都涛声依旧,只要我们消除执念,便可
寂静安然。愿每个人在纷呈世相中不会迷失荒径,可以端坐磐石上,醉倒落花
前。如果可以,请让我预支一段如莲的时光,哪怕将来某一天加倍偿还。这个雨
季会在何时停歇,无从知晓。但我知道,你若安好,便是晴天。
\section{公式}
Soit $f:E\to F$ une fonction  dans $E$. Pour $x\in E$, on a
\[
\int_0^x f'(y)\dd{y}=f(x)-f(0).
\]
Arrays of mathematics are typeset using one of the matrix environments as 
in
\[
\begin{bmatrix}
1 & x & 0 \\
0 & 1 & -1
\end{bmatrix}\begin{bmatrix}
1  \\
y  \\
1
\end{bmatrix}
=\begin{bmatrix}
1+xy  \\
y-1
\end{bmatrix}.
\]
Case statements use cases:
\[
|x|=\begin{cases}
x, & \text{if }x\geq 0\,,  \\
-x, & \text{if }x< 0\,.
\end{cases}
\]
Many arrays have lots of dots all over the place as in
\[
\begin{matrix}
-2 & 1 & 0 & 0 & \cdots & 0  \\
1 & -2 & 1 & 0 & \cdots & 0  \\
0 & 1 & -2 & 1 & \cdots & 0  \\
0 & 0 & 1 & -2 & \ddots & \vdots \\
\vdots & \vdots & \vdots & \ddots & \ddots & 1  \\
0 & 0 & 0 & \cdots & 1 & -2
\end{matrix}
\]


\section{不错的话}
她叫林徽因,出生于杭州,是许多人梦中期待的白莲。她在雨雾之都伦敦,
发生过一场空前绝后的康桥之恋。她爱过三个男子,爱得清醒,也爱得平静。徐
志摩为她徜徉在康桥,深情地等待一场旧梦可以归来。梁思成与她携手走过千山
万水,为完成使命而相约白头。金岳霖为她终身不娶,痴心不改地守候一世。可
她懂得人生飘忽不定,要学会随遇而安。真正的平静,不是避开车马喧嚣,而是
在心中修篱种菊。尽管如流往事,每一天都涛声依旧,只要我们消除执念,便可
寂静安然。愿每个人在纷呈世相中不会迷失荒径,可以端坐磐石上,醉倒落花
前。如果可以,请让我预支一段如莲的时光,哪怕将来某一天加倍偿还。这个雨
季会在何时停歇,无从知晓。但我知道,你若安好,便是晴天。
\section{ et analyse}
\lipsum[30]
\section{Conclusion}
\lipsum[20]
\end{document}
